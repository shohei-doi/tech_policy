% Options for packages loaded elsewhere
\PassOptionsToPackage{unicode}{hyperref}
\PassOptionsToPackage{hyphens}{url}
\PassOptionsToPackage{dvipsnames,svgnames,x11names}{xcolor}
%
\documentclass[
  xelatex,
  ja=standard]{bxjsarticle}

\usepackage{amsmath,amssymb}
\usepackage{iftex}
\ifPDFTeX
  \usepackage[T1]{fontenc}
  \usepackage[utf8]{inputenc}
  \usepackage{textcomp} % provide euro and other symbols
\else % if luatex or xetex
  \usepackage{unicode-math}
  \defaultfontfeatures{Scale=MatchLowercase}
  \defaultfontfeatures[\rmfamily]{Ligatures=TeX,Scale=1}
\fi
\usepackage{lmodern}
\ifPDFTeX\else  
    % xetex/luatex font selection
  \setmainfont[BoldFont=Noto Sans CJK JP]{Noto Serif CJK JP}
\fi
% Use upquote if available, for straight quotes in verbatim environments
\IfFileExists{upquote.sty}{\usepackage{upquote}}{}
\IfFileExists{microtype.sty}{% use microtype if available
  \usepackage[]{microtype}
  \UseMicrotypeSet[protrusion]{basicmath} % disable protrusion for tt fonts
}{}
\makeatletter
\@ifundefined{KOMAClassName}{% if non-KOMA class
  \IfFileExists{parskip.sty}{%
    \usepackage{parskip}
  }{% else
    \setlength{\parindent}{0pt}
    \setlength{\parskip}{6pt plus 2pt minus 1pt}}
}{% if KOMA class
  \KOMAoptions{parskip=half}}
\makeatother
\usepackage{xcolor}
\setlength{\emergencystretch}{3em} % prevent overfull lines
\setcounter{secnumdepth}{5}
% Make \paragraph and \subparagraph free-standing
\ifx\paragraph\undefined\else
  \let\oldparagraph\paragraph
  \renewcommand{\paragraph}[1]{\oldparagraph{#1}\mbox{}}
\fi
\ifx\subparagraph\undefined\else
  \let\oldsubparagraph\subparagraph
  \renewcommand{\subparagraph}[1]{\oldsubparagraph{#1}\mbox{}}
\fi


\providecommand{\tightlist}{%
  \setlength{\itemsep}{0pt}\setlength{\parskip}{0pt}}\usepackage{longtable,booktabs,array}
\usepackage{calc} % for calculating minipage widths
% Correct order of tables after \paragraph or \subparagraph
\usepackage{etoolbox}
\makeatletter
\patchcmd\longtable{\par}{\if@noskipsec\mbox{}\fi\par}{}{}
\makeatother
% Allow footnotes in longtable head/foot
\IfFileExists{footnotehyper.sty}{\usepackage{footnotehyper}}{\usepackage{footnote}}
\makesavenoteenv{longtable}
\usepackage{graphicx}
\makeatletter
\def\maxwidth{\ifdim\Gin@nat@width>\linewidth\linewidth\else\Gin@nat@width\fi}
\def\maxheight{\ifdim\Gin@nat@height>\textheight\textheight\else\Gin@nat@height\fi}
\makeatother
% Scale images if necessary, so that they will not overflow the page
% margins by default, and it is still possible to overwrite the defaults
% using explicit options in \includegraphics[width, height, ...]{}
\setkeys{Gin}{width=\maxwidth,height=\maxheight,keepaspectratio}
% Set default figure placement to htbp
\makeatletter
\def\fps@figure{htbp}
\makeatother

\renewcommand{\thefootnote}{\arabic{footnote}}
\makeatletter
\makeatother
\makeatletter
\makeatother
\makeatletter
\@ifpackageloaded{caption}{}{\usepackage{caption}}
\AtBeginDocument{%
\ifdefined\contentsname
  \renewcommand*\contentsname{目次}
\else
  \newcommand\contentsname{目次}
\fi
\ifdefined\listfigurename
  \renewcommand*\listfigurename{図一覧}
\else
  \newcommand\listfigurename{図一覧}
\fi
\ifdefined\listtablename
  \renewcommand*\listtablename{表一覧}
\else
  \newcommand\listtablename{表一覧}
\fi
\ifdefined\figurename
  \renewcommand*\figurename{図}
\else
  \newcommand\figurename{図}
\fi
\ifdefined\tablename
  \renewcommand*\tablename{表}
\else
  \newcommand\tablename{表}
\fi
}
\@ifpackageloaded{float}{}{\usepackage{float}}
\floatstyle{ruled}
\@ifundefined{c@chapter}{\newfloat{codelisting}{h}{lop}}{\newfloat{codelisting}{h}{lop}[chapter]}
\floatname{codelisting}{コード}
\newcommand*\listoflistings{\listof{codelisting}{コード一覧}}
\makeatother
\makeatletter
\@ifpackageloaded{caption}{}{\usepackage{caption}}
\@ifpackageloaded{subcaption}{}{\usepackage{subcaption}}
\makeatother
\makeatletter
\@ifpackageloaded{tcolorbox}{}{\usepackage[skins,breakable]{tcolorbox}}
\makeatother
\makeatletter
\@ifundefined{shadecolor}{\definecolor{shadecolor}{rgb}{.97, .97, .97}}
\makeatother
\makeatletter
\makeatother
\makeatletter
\makeatother
\ifLuaTeX
\usepackage[bidi=basic]{babel}
\else
\usepackage[bidi=default]{babel}
\fi
\babelprovide[main,import]{japanese}
% get rid of language-specific shorthands (see #6817):
\let\LanguageShortHands\languageshorthands
\def\languageshorthands#1{}
\ifLuaTeX
  \usepackage{selnolig}  % disable illegal ligatures
\fi
\usepackage[]{natbib}
\bibliographystyle{jecon}
\IfFileExists{bookmark.sty}{\usepackage{bookmark}}{\usepackage{hyperref}}
\IfFileExists{xurl.sty}{\usepackage{xurl}}{} % add URL line breaks if available
\urlstyle{same} % disable monospaced font for URLs
\hypersetup{
  pdftitle={はじめに},
  pdfauthor={土井翔平},
  pdflang={ja},
  colorlinks=true,
  linkcolor={NavyBlue},
  filecolor={Maroon},
  citecolor={NavyBlue},
  urlcolor={NavyBlue},
  pdfcreator={LaTeX via pandoc}}

\title{はじめに}
\usepackage{etoolbox}
\makeatletter
\providecommand{\subtitle}[1]{% add subtitle to \maketitle
  \apptocmd{\@title}{\par {\large #1 \par}}{}{}
}
\makeatother
\subtitle{技術政策学(データ科学編)}
\author{土井翔平}
\date{2023-05-29}

\begin{document}
\maketitle
\ifdefined\Shaded\renewenvironment{Shaded}{\begin{tcolorbox}[sharp corners, interior hidden, enhanced, frame hidden, breakable, borderline west={3pt}{0pt}{shadecolor}, boxrule=0pt]}{\end{tcolorbox}}\fi

\href{https://shohei-doi.github.io}{土井翔平}が担当する\href{https://www.hops.hokudai.ac.jp/}{北海道大学公共政策大学院
(HOPS)}の技術政策学(データ科学編)の講義レジュメです。受講生はこのサイトもしくはpdf版を参照して、授業に望んでください。

\hypertarget{ux76eeux6a19}{%
\section{目標}\label{ux76eeux6a19}}

様々な科学技術の中でも近年、発展が目覚ましいデータ科学について学ぶ。\footnote{講師はデータ科学を用いた国際関係の分析を行っている。}

\begin{itemize}
\tightlist
\item
  科学技術のための公共政策:データ科学にはどのような弊害があり、それらに対して社会はどのように対応するべきなのか?
\item
  公共政策のための科学技術:データ科学はにはどのような利点があり、それらをどのように社会のために利活用するべきなのか?
\end{itemize}

\(\leadsto\)代表的なデータ科学の手法について、その概要と長所・短所を学ぶ。

\hypertarget{ux30c8ux30d4ux30c3ux30af}{%
\section{トピック}\label{ux30c8ux30d4ux30c3ux30af}}

データ科学の中でも機械学習と統計的因果推論を扱う。

\begin{itemize}
\tightlist
\item
  \textbf{機械学習}(いわゆる人工知能):多様なデータからパターンを学習し、予測や生成を行う。

  \begin{itemize}
  \tightlist
  \item
    深層学習(ディープラーニング)は機械学習の一手法
  \end{itemize}
\item
  \textbf{統計的因果推論}:人間や社会に関するデータから因果関係、政策効果を推定する。
\end{itemize}

時間が許せば、これらを実際に行うためのプログラミング言語である\href{https://www.python.org/}{Python}を用いた実習を行う。

\begin{itemize}
\tightlist
\item
  ノートPCを持参する。
\item
  Googleアカウントを作成する。

  \begin{itemize}
  \tightlist
  \item
    \href{https://colab.research.google.com/?hl=ja}{Google
    Colaboratory}を使用する。
  \item
    自身のPCにPython環境が構築されている場合は不要
  \end{itemize}
\item
  \href{https://www.mdsc.hokudai.ac.jp/mds/learning-management-system/}{MDSセンター}の教材を用いてPythonの学習をすることを強く推奨する。\footnote{いずれ、\href{https://www.mext.go.jp/a_menu/shotou/zyouhou/detail/1416756.htm}{情報I}を高校で学習した人々が社会に進出してくる。}
\end{itemize}

\hypertarget{ux6388ux696dux306eux9032ux3081ux65b9}{%
\section{授業の進め方}\label{ux6388ux696dux306eux9032ux3081ux65b9}}

Moodleを参照すること。

\hypertarget{ux53c2ux8003ux66f8}{%
\section{参考書}\label{ux53c2ux8003ux66f8}}

\begin{itemize}
\tightlist
\item
  機械学習:\citet{ng2019}, \citet{hisano2018}, \citet{kitagawa2023}
\item
  統計的因果推論:\citet{nakamuro2017}, \citet{ito2017},
  \citet{matsubayashi2021}
\end{itemize}

データ科学については優良な資料がオンラインで無料で公開されているので、各自で調べて参照すること。


  \bibliography{references.bib}


\end{document}
